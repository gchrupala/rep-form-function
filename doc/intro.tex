\begin{abstract}
  We present novel methods for analyzing the activation patterns of
  RNNs from a linguistic point of view and explore the types of 
  linguistic structure they learn. As a case study, we use a multi-task gated
  recurrent network architecture consisting of two parallel pathways with
  shared word embeddings trained on predicting the
  representations of the visual scene corresponding to an input
  sentence, and predicting the next word in the same sentence, 
  in addition to a standard standalone language model. 
  We propose a method for estimating the amount of contribution
  of individual tokens in the input to the final prediction of the networks.   
  Using this method, we show that the image prediction pathway pays
  selective attention to lexical categories and grammatical functions
  that carry semantic information, and also learns to treat word types 
  differently depending on their grammatical function and their position 
  in the sequential structure of the sentence. In contrast, the language 
  models are comparatively more sensitive to words with a syntactic 
  function. Further analysis of the most informative n-gram contexts for 
  each model shows that in comparison to the image prediction pathway,
  the language models react more strongly to abstract contexts 
  that represent syntactic constructions. 
%  Furthermore, we propose methods to explore the function
%  of individual hidden units in RNNs and show that the two pathways
%  of the architecture in our case study contain
%  specialized units tuned to patterns informative
%  for the task, some of which can carry activations to later time
%  steps to encode long-term dependencies.

%Recurrent Neural Network (RNN) models with word embeddings learn
%representations of variable-length linguistic units from text. 
%In the fact that there are no obvious cues to the
%kinds of linguistic information these representations encode lies a vast 
%new area of research questions. The current paper develops general
%methods for the analysis of the learned representations with a focus on
%linguistic \emph{structure} and using \emph{comparative} experiments on two RNNs
%trained on the same data set with different objectives.
%The experiments in the paper consider \emph{macro} and
%\emph{micro} level analysis on the hidden activations of RNNs. On the 
%\emph{macro} level we develop a novel method to measure the saliency of 
%tokens in sentences and show how aggregating 
%these scores in terms of part-of-speech categories
%and dependency relations provides deeper insights into what the hidden
%activations of RNNs learn about linguistic structure. On the \emph{micro}
%level we develop a simple general method to describe
%the functionality of individual hidden units. This technique allows us for the
%exploration of the lingusitic regularities hidden units learn to encode. In
%addition we develop a visualization technique to trace the activations of individual
%units as they carry over their information through multiple time steps. Both on 
%\emph{macro}
%and \emph{micro} level we explore the differences between the examined models
%
%We provide evidence that {\sc Visual}
%learns to treat the same tokens differently depending on their grammatical function.
%In addition we show that it learns to focus on the first Noun in the sentences exploting
%the linear information structure of English. 
 
 
\end{abstract}

\section{Introduction}
\label{sec:intro}
Recurrent neural networks (RNNs) were introduced by
\namecite{elman1990finding} 
as a connectionist architecture with the
ability to model the temporal dimension. They have proved popular for
modeling language data as they learn representations of words and
larger linguistic units directly from the input data, without feature
engineering. Variations of the RNN architectures have been applied in
several NLP domains such as parsing \cite{vinyals2015grammar} and
machine translation \cite{bahdanau2014neural}, as well as in computer
vision applications such as image generation \cite{gregor2015draw} and
object segmentation \cite{visin2015reseg}. RNNs are also important
components of systems integrating Vision and Language, e.g.\ image
\cite{karpathy2015deep} and video captioning \cite{yu2015video}.

These networks can represent variable-length linguistic expressions by
encoding them into a fixed-size low-dimensional vector. The nature and
the role of the components of these representations are not directly
interpretable as they are a complex, non-linear function of the
input. There have recently been numerous efforts to visualize deep
models such as convolutional neural networks in the domain of computer
vision, but much less so for variants of RNNs and for language
processing.
 
The present paper develops novel methods for uncovering abstract
linguistic knowledge encoded by the distributed representations of RNNs,
with a specific focus on analyzing the hidden activation patterns rather 
than word embeddings and on the syntactic generalizations 
that models learn to capture. In the current work we apply our methods
to a specific architecture trained on specific tasks, but also provide
pointers about how to generalize the proposed analysis to other settings.

As our case study we picked the {\sc Imaginet} model introduced by \label{explainimaginet}
\namecite{chrupala2015learning}. It is a multi-task, multi-modal
architecture consisting of two Gated-Recurrent Unit (GRU)
\cite{cho2014properties,chung2014empirical} pathways and
a shared word embedding matrix. One of the GRUs ({\sc Visual}) is
trained to predict image vectors given image descriptions, while the other
pathway ({\sc Textual}) is a language model, trained to sequentially predict each
word in the descriptions. This particular
architecture allows a comparative analysis of the hidden activations
patterns between networks trained on two different tasks, while
keeping the training data and the word embeddings fixed. Recurrent neural
language models akin to {\sc Textual} which are trained to predict the
next symbol in a sequence are relatively well understood, and there
have been some attempts to analyze their internal states e.g.
\cite{elman1991distributed,karpathy2015visualizing}. In
constrast, {\sc Visual} maps a complete sequence of words to
a representation of a corresponding visual scene and is a less
commonly encountered, but a more interesting model from the point of
view of representing meaning conveyed via linguistic structure.
For comparison, we also consider a standard standalone language model.

We report a thorough quantitative analysis to provide a linguistic 
interpretation of the networks' activation patterns. We present a 
series of experiments using a novel method we call \emph{omission score} 
to measure the importance of input tokens to the final prediction of models 
that compute distributed representations of sentences. 
Furthermore, we introduce a more global measure for estimating the 
informativeness of various types of n-gram contexts for each model.
These techniques can be applied to various RNN architectures, Recursive 
Neural Networks and Convolutional Neural Networks. 

Our experiments show that the image prediction pathway in general 
pays special attention to syntactic categories which carry semantic 
content, and particularly nouns. More surprisingly, this pathway also 
learns to treat word types differently depending on their 
grammatical function and their position in the sequential structure of 
the sentence. In contrast, the textual pathway and the standalone 
language model are especially sensitive to the local syntactic 
characteristics of the input sentences.  Further analysis of the most 
informative n-gram contexts for each model shows that while the image 
prediction pathway is mostly sensitive to lexical (i.e., token n-gram) 
contexts, the language models react more strongly to abstract contexts
(i.e., dependency relation n-grams) that represent syntactic constructions. 

%On the macro level, we 
%develop a novel method to estimate the salience of tokens as a function of their syntactic 
%category and grammatical function. We show that due to the nature of its objective, the 
%{\sc Visual} pathway of {\sc Imaginet} learns important information structural properties of 
%language, and treats the same tokens differently depending on their grammatical functions 
%in the sentence. \todo{What can we say about Textual?}

%We also perform a \emph{micro level} analysis to explore \label{edit:introtopkgeneral}
%the function of individual hidden units by introducing a method we dub \emph{top-k-contexts}.
%It involves identifying the sentential contexts which yield the 
%highest activation values and the analysis we present is applicable 
%to uni-directional RNN architectures.
%Through a qualitative examination of these contexts we
%observe that in case of {\sc Imaginet} some of the contexts represent a particular 
%syntactic category or dependency function, while others correspond to a semantic theme. 
%We identify cases where the hidden units encode characteristics of 
%the context that go beyond their lexical properties, and represent abstract 
%patterns that are useful for the network's task. 
%Furthermore, we explore and visualize units that carry their
%activation over to later time steps to 
%extract longer dependencies and more complex linguistic features. 
%We also quantitatively show that features encoded by the language model are more associated 
%with syntactic constructions than in case of the image prediction pathway.

%\todo{No mutual information results. should we say something about that?}
%
%Lastly, we perform a comparative analysis between {\sc Visual} and {\sc Textual} and 
%present quantitative evidence that the features encoded by the hidden units of {\sc 
%Textual} are more focused on syntactic patterns than in case of {\sc Visual}.
%\todo{Not sure about this last line...}


%In all our experiments we use fixed RNN architecture with Gated Recurrent Units (GRU) 
% and train them on there tasks: 1) sentiment analysis ;
% 2) language modelling ; and 3) image prediction (predicting the visual 
% representation of an input word sequences). These 
%three tasks require different features to be extracted from
%linguistic data, which forces us to develop methods that generalize
%to a variety of models.  

%Recurrent Neural Networks (RNNs) were introduced by 
%\cite{elman1990finding} as a connectionist architecture with the ability to model the temporal
%dimension. This feature makes RNNs particularly suitable for modelling 
%tasks that process input sequences, such as a variety of language 
%processing tasks. \cite{elman1991distributed} himself investigated the nature of the
%distributed representations built by a simple recurrent network when
%reading sentences and learning to predict each coming word, and showed 
%that the network grasped some notion of lexical categories among the 
%(very) small set of words in the input which show a syntactically similar 
%behaviour (e.g., edible or inanimate objects).  


%RNNs can usually detect and generalise structural information that is implicitly present 
%in the sequence data and is useful for the task they are optimised for; however, the exact 
%nature of the structural  knowledge learned by RNNs is hard to pinpoint. 


%With the recent surge of interest in artificial neural networks comes the 
%need for techniques for analysing the internal dynamics as well as the 
%observable behaviour of such networks in order to understand the type of 
%structural, higher-level knowledge they extract and use from the input data. 
%Some efforts have been made in the domain of computer vision: .... 
%\todo[inline]{guys, please summarise the current approaches in a couple of 
%sentences.} 

%Similar studies are needed in the NLP domain, where the emergence of 
%syntactic knowledge from sequence of words is of special interest. The 
%extent and complexity of the acquired syntactic knowledge depends on the 
%nature of the task a model is trained for. In this paper we explore what 
%linguistic features RNNs learn from unannotated word sequences in order 
%to solve tasks that involve different levels of natural language 
%understanding. More specifically, we are interested in 
%\begin{itemize}
%We focus on different levels of describing the models as well as different 
%types of the tasks these models solve. The experiments explore:
%\item the nature of the syntactic structure the RNNs learn;
%\item the syntactic categories of words different models pay most attention to;
%\item the specific function of (some of) the individual neurons in these 
%models;
%\item the level of separation between syntactic and semantic 
%representation.
%\end{itemize}

%In this study we follow in this tradition but take onboard all the recent
%developments in recurrent architectures and their application in a
%variety of domains. Specifically, w



%We work with gated units able to
%learn long dependencies, and in addition to language modeling, we
%consider other tasks such as predicting sentiment and predicting
%features of the visual scene corresponding to a sentence.  Based on
%our experiments we show that depending on the nature of the prediction
%task on which a network is trained, it learns to encode either (a)
%aspects of the syntactic structure of the sentence, or (b) aspects of
%its meaning: concretely, a language modeling network will be
%comparatively more sensitive to function words encoding syntactic
%information, while a network predicting visual features pays most
%attention to semantically rich content words. Additionally we also
%show how the representations built by recurrent neural network evolve
%over time while incorporating each coming word: syntactically parallel
%and semantically related sentences trace similar trajectories in the
%vector space of the hidden layer states.

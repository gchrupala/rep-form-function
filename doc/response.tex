\newpage
\pagenumbering{roman}

\noindent
Dear Dr.~Merlo,
\newline

Thank you for the very helpful suggestions and insightful questions
posed in this review.  We apologize for the delay getting this
revision back to you and very much appreciate the extension you 
gave us to resubmit the manuscript.

We have substantially revised the article in response to your and the reviewers' 
comments. Specifically with regard to the two main issues you raised, we have 
introduced a uni-modal language model (referred to as the {\sc LM} model in the text) 
and reported the results of all the experiments on this model as well. Also following 
your recommendation, we have removed the exploratory parts of (the old) Section 6, 
and reorganized the paper by merging previous Sections 4, 5 and the remaining of 
Section 6 into a single section (Section 4 in the revised manuscript) containing all the 
experiments. We have kept and expanded the quantitative analysis of the nature of 
n-gram contexts that trigger individual dimensions in the hidden layers; this experiment is now reported in Section~\ref{sec:contexts}.

Furthermore, we have made clarifications about the details of the techniques all 
through the manuscript, expanded Introduction (Section~\ref{sec:intro}) to 
emphasise our contributions, updated Related Work (Section~\ref{sec:related}) 
to include relevant papers that came out while this manuscript was under review, and expanded the Discussion to discuss in more detail the relevance and generalizability of our techniques. 

We believe that the revisions to the paper address your collective
concerns, and we thank you all again for the comments that have helped
us to improve the paper.
\newline

\noindent
Sincerely,
\newline

\noindent
Ákos Kádár, Grzegorz Chrupała and Afra Alishahi
\newline


\begin{verbatim}
Dear Ákos Kádár:

We have reached a decision regarding your submission to Computational
Linguistics, "Representation of linguistic form and function in
recurrent neural networks". I include here the reviews for your
perusal.

The reviewers and I like the area of work in which this paper is
situated: model analysis. While the reviews correctly point out that
several shortcomings remain, a discussion has identified the
modifications that we think are really required, to make the paper
stronger. Other ideas are suggestions that we leave to you for future
work.

The requested changes concern an added comparisons to unimodal models
(or at least to unimodal versions of Imaginet). On the other hand, all
the reviewers thought that Section 6 is less convincing than the rest
of the paper, at least in the current writeup. So we think that this
Section could be reduced in favor of adding the unimodal experiments .

I would be grateful if you could provide an appropriately revised
version of your paper by 10th February 2017. Please email the
editorial assistant (editorialoffice@cljournal.org) within one week,
to confirm whether you will be able to meet this target date, or to
negotiate a new target date.

You will find details of the resubmission process at
http://cljournal.org/ojs-help.html.  When you submit your revised
paper, please enclose a letter that responds specifically to each of
the points made by the reviewers, outlining the changes you have made,
and those you have not addressed, together with an electronic copy of
the revised manuscript as a PDF file.

Thank you for submitting to Computational Linguistics.


Paola Merlo
University of Geneva
Paola.Merlo@unige.ch
------------------------------------------------------
Reviewer A:


2 What is this paper about? [Help] :
The paper presents methodologies for analyzing what is learned by
recurrent neural networks. It does this by suggesting two
methodologies, and then applying them to two similar tasks -- one that
that is based on language modeling, and another which is based on
visual recognition. The analysis highlights some of the things that
are encoded and not encoded in each of the representations.


3 Strengths and Weaknesses [Help]

Strengths:

The topic of understanding what is learned by neural network models,
in particular recurrent ones, is of great importance and interest, and
papers of this kind are very much needed. While most work tries to use
"deep learning" and RNNs to "improve" on various language tasks,
understanding their capabilities and inner working, and making them
more transparent, is much more challenging, and there are not enough
works in this area. The current work attempts to do precisely that,
and it also delivers on its promise with an interesting methodology
and insights. In particular, the first part (section 5) is
particularly well done. Section 6 is somewhat less convincing, but
still is on-par or better than any other work that attempts to deliver
analysis and insights.


Weaknesses:
 
Some parts of the papers were not clear, and could be improved. I also
have someissues with some of the experiments in section 6, which can
either be presented better or even removed altogether, in my view.

- In terms of clarity, I would appreciate a better description of the
  IMAGINET model, which is being analyzed. The model is described in
  terms of its equations, but I would like to see a better
  presentation of what is the intent behind it -

- what is being modeled and what is being learned? this is available
  in the original IMAGINET paper, but I think the description should
  be imported to this paper also, in order to make it self contained
  and stronger.
  
\end{verbatim}  
\begin{quote}
\textsc{Author Response:}
We have expanded Section~\ref{sec:imaginet} with a conceptual
description of {\sc Imaginet} and its learning goals, and added a graphical
representation of the structure of the model for more clarity (Figure~\ref{fig:imaginet}).
\end{quote}
\begin{verbatim}

- Also in terms of clarity, the graphs are not described in enough
  detail in my view. Whiel the box plots (figure 2 and others) are
  somewhat standard, they are not mainstream in the NLP commuity, and
  I would appreciate seeing a more detailed description in the main
  text or the caption of what each component means, both generally and
  in the context of the specific graph.
\end{verbatim}  
\begin{quote}
\textsc{Author Response:}  \todo[inline]{explain the updated graph}
\end{quote}
\begin{verbatim}
- Figure 7 is confusing -- what are the different colors between,
  e.g., the words "in" and "tree" in the first line, and between "a"
  and "wii" in the second? I assume this is some "gradient" between
  two colors, but it looks like there are several data points and is
  very confusing. Please fix this.
\end{verbatim}  
\begin{quote}
\textsc{Author Response:}  Following the recommendation of the reviewers,
we shortened Section~\ref{sec:micro} and removed this figure.
\end{quote}
\begin{verbatim}

In terms of content:

- In section 6.1, why is the sorting by magnitude and not by absolute
  value? are negative activations not important?
\end{verbatim}  
\begin{quote}
\textsc{Author Response:}  It is in fact the absolute value of the activations 
that is taken into account. We have changed the description of the top-k-contexts 
method to clear this issue.
\end{quote}
\begin{verbatim}

- When choosing the "top k context of each unit" (section 6.1), it is
  not clear what a "unit" is and how it is chosen.
\end{verbatim}  
\begin{quote}
\textsc{Author Response:}  A unit is one neuron in the final hidden layer 
of the network. We have made this clear in the revised text.
\end{quote}
\begin{verbatim}

- I did not understand the last sentence in section 6.1, and would
  appreciate an elaboration on this point (or a removal).
\end{verbatim}  
\begin{quote}
\textsc{Author Response:}  We removed that sentence and added an expanded discussion of the generalizability of our techniques to the last section.
\end{quote}
\begin{verbatim}

- In section 6.2, it is not clear how trigrams with high activations
  are chosen. How is a trigram with high activation defined? this is
  not clear from the description of the method.
\end{verbatim}  
\begin{quote}
\textsc{Author Response:} We removed this section in the revised manuscript. We address the nature of contexts which trigger individual dimensions in more detail now in Section~\ref{sec:contexts}.
\end{quote}
\begin{verbatim}

- I am not convinced by the experiments in section 6.3 and their
  importance -- either they are not described properly and I missed
  something, or they are somewhat meaningless: when training a
  logistic regression model to predict grammatical function, it is not
  surprising that ones finds units that are predictive of grammatical
  function.
\end{verbatim}  
\begin{quote}
\textsc{Author Response:}  We have reorganized Section~\ref{sec:micro} and 
tried to motivate this particular experiment better. The idea is to 
come up with techniques for singling out units with a specific type
of function, such as encoding grammatical function. Therefore we 
use the coefficients of the logistic regression model as pointers 
to units of this type.
\end{quote}
\begin{verbatim}


~ Where is section 6.4?

\end{verbatim}  
\begin{quote}
\textsc{Author Response:} It was there, believe us! But we have reorganized 
Section~\ref{sec:micro} and removed this experiment for the sake of clarity.
\end{quote}
\begin{verbatim}

~ a related (unpublished, but available on arxiv) work is
https://arxiv.org/abs/1608.04207

\end{verbatim}  
\begin{quote}
\textsc{Author Response:}  We have expanded the discussion of related work 
(Section~\ref{sec:related}) and included this work in addition to a few others which 
came out while our submission was under review.
\end{quote}
\begin{verbatim}

4 Substantive Revisions Required [Help]

Complete this section if either 'Revise and Resubmit' or 'Reject' has
been recommended.

Revisions to be Required:
: 



Revisions to be Encouraged:
: 



5 Minor Revisions Required [Help]
:

See the points marked with (-) in the weaknesses above. Points marked
with (~) are less important. For section 6.3, my recommendation is
that it is either substantially revised, or removed altogether -- I
believe the paper has enough merit without it.

\end{verbatim}  
\begin{quote}
\textsc{Author Response:}  We hope we have addressed all your comments and concerned. We did not remove the experiments in Section 6.3, but as we mentioned before we rewrote and restructured Section~\ref{sec:micro} and tried to put this experiment in context and clear our motivation.
\end{quote}
\begin{verbatim}

6 Typographic Errors [Help]
:

- "by as" in the first sentence of the intro.

- There are some cases of using parenthesis in citation where they
  shouldn't be, for example the first two citations in section 2 ("of
  (Elman 1990)" should be "of Elman (1990)").

\end{verbatim}  
\begin{quote}
\textsc{Author Response:} the typo and the citation inconsistencies are fixed.
\end{quote}
\begin{verbatim}

------------------------------------------------------

------------------------------------------------------
Reviewer E:


2 What is this paper about? [Help]
: 

The paper presents a set of methods for analyzing the activations of
recurrent neural networks from a linguistic perspective using:

a) omission scores (to examine the contribution of a single token to
the prediction of the network); and

b) top k contexts (keeping the activations for the entire sequence and
examining individual dimensions)

The model used in this paper is a multimodal (text and visual)
multi-task RNN (Imaginet). At each time step the next word is
predicted from its current state, and an image vector is predicted
from the model's final state. The model consists of two parallel GRUs
with shared word embeddings.

By pairing omission scores with POS and dependency relations, the
authors show that the Visual pathway mostly focuses on noun and noun
relations, while the textual pathway focuses more on function
words. The textual pathway has in general a more uniform omission
score distribution, the visual model peaks on content words (albeit
not particularly on verbs, as the authors discuss).


3 Strengths and Weaknesses [Help]

Strengths:
: 
* neat idea and novel contribution (pairing activations with POS and
dependency relations to gain further insights into RNNs)

* extensive analysis; especially going beyond lexical cues (Section
5.3), where the authors train a regression model to predict omission
scores on a token level, showing that the network not just learns to
rely on word types, but it also learns position information
(particularly important for the textual pathway)


Weaknesses:
: 
* the description of the top K contexts is not clear, how are top k
contexts calculated? do you find the most active token (highest
absolute value of hidden activation in the sequence?) and then use
the words around it to find the trigrams? (what is meant by 'each
unit' in Section6.1? Is the matrix actually R^nxd?); maybe you sort
by column, not row? 

\end{verbatim}  
\begin{quote}
\textsc{Author Response:}  You are absolutely right about the description of 
the top-k-context method being confusing. We have rewritten this part to make the description of the method more clear. Furthermore, we have reorganized 
Section~\ref{sec:micro} to make it more coherent. 
\end{quote}
\begin{verbatim}

* Section 6.5 shows that the visual pathway focuses more on dependency
relations, supporting the finding in the omission score part;
however, the section is again very dense and Figure 7 only shows
activations for Visual; I'd suggest to drop this part from the paper

\end{verbatim}  
\begin{quote}
\textsc{Author Response:}  Following your and others' suggestion, we have removed this section from the revised manuscript.
\end{quote}
\begin{verbatim}

4 Substantive Revisions Required [Help]

Complete this section if either 'Revise and Resubmit' or 'Reject' has
been recommended.

Revisions to be Required:
: 
- clearer description of how top k contexts are obtained
\end{verbatim}  
\begin{quote}
\textsc{Author Response:}  As mentioned above, we have rewritten Section~\ref{sec:micro} to make the description of the methods and the experiments more clear.
\end{quote}
\begin{verbatim}


Revisions to be Encouraged:
: 
- figures are not readable in BW printing; in addition to color, use
  different symbols/line types
\end{verbatim}  
\begin{quote}
\textsc{Author Response:}  \todo[inline]{update graphs?}
\end{quote}
\begin{verbatim}

- Related Work: clarify better how you differ from Li et al. (2015),
  you mention "for a comparative analysis": I interpret this as "we
  analyze a multi-task learning model (visual / textual) and compare
  what the two parts learn"
\end{verbatim}  
\begin{quote}
\textsc{Author Response:} We have edited our related work section to make the comparison more 
clear. In addition to taking into account tasks that rely on different modalities (language as well as vision), 
we focus more on learning structural properties of the input, whereas Li et al. (2015) focus on single
tokens.
\end{quote}
\begin{verbatim}


5 Minor Revisions Required [Help]
: 
- Sec 2, clarify: what are 'negative strings'? (non-words?)
\end{verbatim}  
\begin{quote}
\textsc{Author Response:}  we have reworded `positive' and `negative' as `grammatical' 
and `ungrammatical' to avoid confusion.
\end{quote}
\begin{verbatim}
- Before equation 9 missing lambda ("weighted by \lambda"; what lambda
  is used in the exp.?)
\end{verbatim}  
\begin{quote}
\textsc{Author Response:}  There was no $\lambda$ in that equation, but in 
Section{sec:imaginer} we now report the value we use for the parameter 
$\alpha$ in Equation~(\ref{eq:losscombo}).
\end{quote}
\begin{verbatim}
- it would be helpful to include a figure of model (as in original
  paper)
\end{verbatim}  
\begin{quote}
\textsc{Author Response:}  We have included an illustration of the structure of the model in Figure~\ref{fig:imaginet}.
\end{quote}
\begin{verbatim}
- "two minor modifications" but the first seems you also use cosine,
  was that different previously, not clear.
\end{verbatim}  
\begin{quote}
\textsc{Author Response:}  You are right, the description of the model in Section~\ref{sec:imaginet} presents the version used in this paper, which uses cosine distance. The original Imaginet model uses mean square error instead; we clarified this issue in footnote~\ref{ft:imaginet}.  
\end{quote}
\begin{verbatim}
- specify interaction features in regression model
\end{verbatim}  
\begin{quote}
\textsc{Author Response:}  We have specified the interaction features used in each regression model in Section~\ref{sec:beyond lexical}.
\end{quote}
\begin{verbatim}
- 5.3.1. why also 'water'?
\end{verbatim}  
\begin{quote}
\textsc{Author Response:}  We have removed the {\it water} from the set of examples in Figure~\ref{fig:top_words}.
\end{quote}
\begin{verbatim}


6 Typographic Errors [Help]
: 
- abstract: sentence "Based on.." way too long
- mention multi-task in abstract
- introduction: "this includes various" > "This can be applied to
  various"
- from linguistic point (missing det)
- generally avoid long sentences (e.g., last paragraph sec 5.1,
  sec. 5.2)
- rephrase: "this model we refer to as SUM"
- windowsizes
\end{verbatim}  
\begin{quote}
\textsc{Author Response:}  we have fixed these issues, thanks for bringing them to our attention.
\end{quote}
\begin{verbatim}

------------------------------------------------------

------------------------------------------------------
Reviewer F:


2 What is this paper about? [Help]
: 
This paper presents an analysis of the information learned by
recurrent neural networks. It focuses on one particular multimodal,
multitask model, Imaginet, which is trained on images and their
descriptions. The paper proposes a new measure, the omission score,
which indicates how sensitive the network is to individual words in
the input. Using this method, the authors show that the textual module
of the model pays attention to function words, while the visual module
pays attention to content words. The authors then propose a second
method which can be used to analyze which n-grams the hidden units of
the model are sensitive to; this results in meaningful syntactic and
semantic clusters. This time, there is no clear-cut distinction
between textual and visual modules.


3 Strengths and Weaknesses [Help]

Strengths:
: 
This paper makes a useful methodological contribution by proposing
methods to analyze the knowledge learned by neural networks, which is
an underexplored area, at least for networks trained on text data. The
paper is well written, and technically sound.


Weaknesses:
: 
The results of the paper are unfortunately fairly obvious. From the
omission score analysis, we learn that a model trained to predict the
next word in a sentence is sensitive to function words, while a word
trained to associate a sentence with an image representation (which is
essentially the output of the object detector) pays more attention to
nouns. This is not at all surprising, it just reflects the different
training objectives. The fact that the hidden units respond to
specific n-grams is also expected; most people already believe that
RNNs are good at approximating n-gram models; the question is, can
they model more elaborate syntactic structures?
\end{verbatim}  
\begin{quote}
\textsc{Author Response:}  \todo[inline]{address}
\end{quote}
\begin{verbatim}

This is really the question that the authors should explore, e.g., by
testing whether the model can learn long distance dependencies,
agreement, recursion, word order. An example for recent work that goes
in this direction is Linzen et al. (2016).
\end{verbatim}  
\begin{quote}
\textsc{Author Response:} We have expanded the related work 
(Section~\ref{sec:related}) with a discussion of Linzen et al. (2016) 
and a few other articles that came out while our submission was under review.
\end{quote}
\begin{verbatim}

Another limitation of the paper is its focus on one particular model,
which happens to be the authors'. To be convincing, the paper would
really need to compare results for this model with results for other
architectures, showing that the proposed analysis methods offer real
discriminative power.
\end{verbatim}  
\begin{quote}
\textsc{Author Response:}  Due to space and time limitations we have to 
leave the comparison with other architectures for future work; however we 
have added a section to Discussion to discuss the generalizability of our 
techniques to other architectures in more detail.
\end{quote}
\begin{verbatim}

Related to this, it is not clear to me why a multimodal, multitask
model such as Imaginet is required for the work presented
here. Surely, similar results would be obtained if two unimodal RNN
models were trained, one on image descriptions to predict the next
word, and one on image representations to predict the description.
\end{verbatim}  
\begin{quote}
\textsc{Author Response:}  \todo[inline]{address}
\end{quote}
\begin{verbatim}

4 Substantive Revisions Required [Help]

Complete this section if either 'Revise and Resubmit' or 'Reject' has
been recommended.

Revisions to be Required:
: 
(1) Add comparisons to unimodal models (or at least to unimodal
versions of Imaginet).
\end{verbatim}  
\begin{quote}
\textsc{Author Response:}  In the revised manuscript we have introduced a uni-modal language model, and updated all are experiments to include this model as well.
\end{quote}
\begin{verbatim}

(2) Add comparisons to other architectures (LSTMs, etc.)
\end{verbatim}  
\begin{quote}
\textsc{Author Response:}  As we mentioned before, we are very interested in 
applying our proposed techniques to other tasks and architectures in future work. 
For the sake of this manuscript, we have included a theoretical discussion of how these techniques can be implemented for a number of other architectures.
\end{quote}
\begin{verbatim}

(3) Sec. 5.3.2 should be cut. You don't have connected discourse, just
isolated sentences. So it is purely speculative to say these
observations are due to topic/content structure, rather than just to
linear order.
\end{verbatim}  
\begin{quote}
\textsc{Author Response:}  Our explanation of the aim of this experiment in the original version might have been misleading; we have rewritten this section to make it clear that we are examining the information structure within a single sentence and not in a dialogue. We do believe that the results we report in this section are interesting and nontrivial, especially the fact that the position usually occupied by the subject of the sentence is the most important for the {\sc Visual} model despite the common recency bias in RNNs.
\end{quote}
\begin{verbatim}

(4) The paper relies too much on impressionistic evaluation based on
suggestive figures, rather than on quantitative analysis. An exception
is Fig. 8. There should be more of this type of analysis.
\end{verbatim}  
\begin{quote}
\textsc{Author Response:}  We have removed the exploratory parts of (the old) Section 6, and expanded the quantitative analysis of the nature of triggering contexts in Section~\ref{sec:contexts}. 
\end{quote}
\begin{verbatim}

(5) The authors should discuss the recent paper by Linzen et
al. (2016). This is a good example for an attempt to evaluate
linguistic structure learned by an RNN that goes beyond individual
words or n-grams.
\end{verbatim}  
\begin{quote}
\textsc{Author Response:}  As mentioned before, we have included a discussion of
this paper in our related work.
\end{quote}
\begin{verbatim}

Reference

Assessing the Ability of LSTMs to Learn Syntax-Sensitive
Dependencies. Tal Linzen, Emmanuel Dupoux, Yoav Goldberg. To appear in
TACL 2016. https://arxiv.org/abs/1611.01368v1


Revisions to be Encouraged:
: 



5 Minor Revisions Required [Help]
: 



6 Typographic Errors [Help]
: 


------------------------------------------------------
\end{verbatim}
